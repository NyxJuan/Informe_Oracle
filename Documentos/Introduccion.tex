\section{INTRODUCCION}
\begin{enumerate}
\vspace{12pt}

Oracle es básicamente un herramienta cliente/servidor para la gestión de base de datos, es un producto vendido a nivel mundial, aunque la gran potencia que tiene y su elevado precio hace que solo se vea en empresas muy grandes y multinacionales, por norma general.\\

En el desarrollo de paginas Web pasa lo mismo como es un sistema muy caro no está tan extendido como otras bases de datos, por ejemplo, Access, MySQL, SQL Server etc.\\


Oracle como antes lo mencionamos se basa en la tecnología cliente/ servidor, pues bien, para su utilización primero seria necesario la instalación de la herramienta servidor ( Oracle8i ) y posteriormente podríamos atacar a la base de datos desde otros equipos con herramientas de desarrollo como Oracle Designer y Oracle Developer, que son las herramientas de programación sobre Oracle a partir de esta premisa vamos a desarrollar las principales acepciones de Oracle y sus aplicaciones en las distintas ares de trabajo.\\

Es posible lógicamente atacar a la base de datos a través del SQL plus incorporado en el paquete de programas Oracle para poder realizar consultas, utilizando el lenguaje SQL.\\

El Developer es una herramienta que nos permite crear formularios en local, es decir, mediante esta herramienta nosotros podemos crear formularios, compilarlos y ejecutarlos, pero si queremos que los otros trabajen sobre este formulario deberemos copiarlo regularmente en una carpeta compartida para todos, de modo que, cuando quieran realizar un cambio, deberán copiarlo de dicha carpeta y luego volverlo a subir a la carpeta.\\

Este sistema como podemos observar es bastante engorroso y poco fiable pues es bastante normal que las versiones se pierdan y se machaquen con frecuencia. La principal ventaja de esta herramienta es que es bastante intuitiva y dispone de un modo que nos permite componer el formulario, tal y como lo haríamos en Visual Basic o en Visual C, esto es muy de agradecer.\\

\end{enumerate}
